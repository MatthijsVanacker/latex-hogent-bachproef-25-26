%==============================================================================
% Sjabloon onderzoeksvoorstel bachproef
%==============================================================================
% Gebaseerd op document class `hogent-article'
% zie <https://github.com/HoGentTIN/latex-hogent-article>

% Voor een voorstel in het Engels: voeg de documentclass-optie [english] toe.
% Let op: kan enkel na toestemming van de bachelorproefcoördinator!
\documentclass{hogent-article}

% Invoegen bibliografiebestand
\addbibresource{voorstel.bib}

% Informatie over de opleiding, het vak en soort opdracht
\studyprogramme{Professionele bachelor toegepaste informatica}
\course{Bachelorproef}
\assignmenttype{Onderzoeksvoorstel}
% Voor een voorstel in het Engels, haal de volgende 3 regels uit commentaar
% \studyprogramme{Bachelor of applied information technology}
% \course{Bachelor thesis}
% \assignmenttype{Research proposal}

\academicyear{2025-2026} % TODO: pas het academiejaar aan

% TODO: Werktitel
\title{Datavisualisatie in sport analytics: een proof-of-concept voor trainingsinzicht in voetbal}

% TODO: Studentnaam en emailadres invullen
\author{Matthijs Vanacker}
\email{matthijs.vanacker@student.hogent.be}

\supervisor[Co-promotor]{J. Doe (Company, \href{mailto:john.doe@company.be}{john.doe@company.be})}

\specialisation{AI \& Data Engineering}
\keywords{Sport, Data analyse, Data behandeling}

\begin{document}

\begin{abstract}
  Er wordt als maar vaker data verzameling en data analyse verricht in de sportwereld, zoals bij voetbal, echter benutten coaches en clubs deze data vaak niet optimaal. Een reden hiervoor is de manier waarop deze opgehaalde data bij de coach terecht komt, deze manier zorgt ervoor dat deze niet altijd de juiste inzichten krijgt en dus ook niet altijd de juiste beslissingen kan maken. Zo luidt de centrale onderzoeksvraag: Hoe kunnen we opgehaalde trainingsdata omzetten in een beter inzicht in trainingsprestaties? Het doel van dit onderzoek is een proof-of-concept maken die de data duidelijk, leesbaar en interpreteerbaar visualiseert om zo de kloof tussen het ophalen en gebruiken van de data te verkleinen. 
  Na de succesvolle uitvoering van deze bachelorproef is er aangetoond dat een betere visualisatie van de trainingsdata positieve gevolgen kan hebben met betrekking tot beslissingen nemen op training en/of in wedstrijden. Dit onderzoek is specifiek gericht op voetbal maar de resultaten van dit onderzoek zijn mogelijks ook door te trekken naar vele andere disciplines.
\end{abstract}

\tableofcontents

% De hoofdtekst van het voorstel zit in een apart bestand, zodat het makkelijk
% kan opgenomen worden in de bijlagen van de bachelorproef zelf.
%---------- Inleiding ---------------------------------------------------------


\section{Inleiding}%
\label{sec:inleiding}

In de sportwereld, zoals bij voetbalclubs, is het primaire doel om vooruitgang te boeken. 
Hiervoor wordt zo veel mogelijk gebruik gemaakt van de recentste, beste technologieën zoals wearables, GPS-trackers en videoanalyse. 
Deze technologieën hebben het potentieel om invloed te hebben op allerlei beslissingen die moeten worden genomen, er blijkt echter dat coaches, ondanks het bezitten van deze waardevolle data, nog steeds vaak hun beslissingen laten beïnvloeden door andere factoren. 
Dit komt wellicht omdat de kloof tussen het ontvangen van de data en het benutten ervan nog te groot is.

Deze bachelorproef focust zich op een specifieke casus binnen een voetbalcontext, waar men reeds veel data verzamelt. 
De doelgroep van het onderzoek bestaat uit coaches en performance analisten binnen de voetbalwereld, die nut hebben aan duidelijke, snel interpreteerbare inzichten tijdens trainingssessies.
In samenwerking met een club zal duidelijker worden welke data exact wordt verzameld en hoe deze zo interpreteerbaar kan gemaakt worden. 
Mogelijke data is de data die opgehaald wordt door de \emph{Catapult}-wearables, die gebruikt worden door vele Belgische clubs.
\newline\newline
De centrale onderzoeksvraag van deze bachelorproef luidt: Hoe kunnen we opgehaalde trainingsdata omzetten in een beter inzicht in trainingsprestaties?
Het onderzoek heeft de doelstelling een oplossing te vinden die de kloof tussen het ophalen en toepassen van de trainingsdata te verkleinen. 
Concreet is het doel een proof-of-concept te ontwikkelen dat aantoont dat trainingsdata beter te visualiseren is om zo de eerder genoemde kloof te dichten.

Enkele deelvragen met betrekking tot het probleemdomein zijn: Welke trainingsdata wordt reeds opgehaald? Hoe wordt deze trainingsdata opgehaald? Wat doen coaches met de opgehaalde trainingsdata? Hoe vaak beïnvloedt de trainingsdata keuzes?

Enkele deelvragen met betrekking tot het oplossingsdomein zijn: Welke types data worden opgehaald? Hoeveel tijd kost het om de data te verzamelen? In welke mate helpt visualisatie de coach om de data te interpreteren? Op welke manier wordt de data gevisualiseerd?

%---------- Stand van zaken ---------------------------------------------------

\section{Stand van zaken}%
\label{sec:literatuurstudie}

In de sportwereld heeft de digitalisering de afgelopen jaren gezorgd voor een toename in het gebruik van diverse technologieën, zoals wearables, GPS-trackers en data-analyseplatformen, bij trainingen en wedstrijden. Hierdoor worden er in allerlei sportdisciplines grote hoeveelheden data verzameld om de performantie van atleten beter te begrijpen en te kunnen verbeteren. Hoewel deze technologieën nu al door veel ploegen en individuën gebruikt worden, tonen recente onderzoeken aan dat deze opgehaalde data vaak slechts sporadisch wordt toegepast en dat coaches moeite ervaren om de gegevens om te zetten naar praktische bruikbare inzichten.

Uit het onderzoek van \textcite{AnyadikeDanesEtAl2023} blijkt dat clubs en coaches wel toegang hebben tot deze data, maar dat andere factoren vaak een grotere rol spelen. Zo spelen niet-fysieke factoren, zoals de mentaliteit van een speler, quasi altijd een rol om een beslissing te nemen en is de fysieke prestatie slechts volgens een vierde van de participanten de hoofdreden om een beslissing te nemen.

De relatie tussen een coach en zijn analisten is ook cruciaal bij het toepassen van de moderne technologieën volgens \textcite{CallinanEtAl2023}. Volgens deze studie is dit omdat coaches vaak onrechtstreeks, via analisten of \emph{generalisten}, in contact staan met de opgehaalde data. Dit zorgt dus ook voor een extra moeilijkheid bij een coach omtrent het inzichten halen uit de opgehaalde data. Dit onderzoek was specifiek naar rugby-ploegen, maar aangezien de meeste professionele sportverenigingen tegenwoordig beschikken over analisten, is er reden om te geloven dat de conclusie uit dit onderzoek ook toepasselijk is op een bredere schaal.

De onderzoeken van zowel \textcite{HoutmeyersEtAl2021}, \textcite{ObiEtAl2024} en \textcite{DavisEtAl2024} tonen allen de impact die data analyse op sport heeft en kan hebben. Dit benadrukt dat het absoluut waardevol kan zijn om de kloof te dichten.



% Voor literatuurverwijzingen zijn er twee belangrijke commando's:
% \autocite{KEY} => (Auteur, jaartal) Gebruik dit als de naam van de auteur
%   geen onderdeel is van de zin.
% \textcite{KEY} => Auteur (jaartal)  Gebruik dit als de auteursnaam wel een
%   functie heeft in de zin (bv. ``Uit onderzoek door Doll & Hill (1954) bleek
%   ...'')


%---------- Methodologie ------------------------------------------------------
\section{Methodologie}%
\label{sec:methodologie}

De eerste fase van het onderzoek zal zijn om in de eerste weken door middel van een literatuurstudie en interviews bij clubs te onderzoeken hoe de huidige dataverzameling en -interpretatie verloopt, welke data wordt opgehaald, wat wordt hiermee gedaan\dots.
Hierna volgt een korte periode waarbij er gezocht wordt naar een haalbare manier om de huidige methode te verbeteren.
Wanneer deze methode uitgedacht is, begint de effectieve ontwikkeling van het proof-of-concept. Hier wordt dan een basische versie ontwikkeld van een programma of dashboard of andere uitgedachte methode die helpt om de gevisualiseerde data leesbaar over te brengen aan de coach van de club. Hierbij zullen ook de behoeftes van coaches worden opgenomen om het protoype zo goed mogelijk te maken.
Nadat het proof-of-concept gemaakt is volgt nogmaals een gesprek met de belanghebbenden binnen de club, om af te stemmen of het product voldoet aan hun verwachtingen.

%---------- Verwachte resultaten ----------------------------------------------
\section{Verwacht resultaat, conclusie}%
\label{sec:verwachte_resultaten}

Ik verwacht dat het duidelijk en leesbaar visualiseren van de trainingsdata coaches in staat stelt om beter beslissingen te nemen op vlak van deze data, in vergelijking met onrechtstreeks contact met de data, of bij onduidelijke representaties hiervan. De niet-fysieke factoren zullen ook nog een rol blijven spelen maar met dit onderzoek wil ik aantonen dat het maken van de beslissingen op basis van de fysieke factoren nog beter en makkelijker wordt.
Het onderzoek zal aantonen dat het probleem niet de beschikbaarheid van data is, maar de manier waarop de coach deze data krijgt te zien. Daarom moet de proof-of-concept aantonen dat een goede visualisatie van de data leidt tot het verkleinen van de kloof tussen dataverzameling en -toepassing in de sportwereld.

\printbibliography[heading=bibintoc]

\end{document}